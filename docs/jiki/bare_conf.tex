
%% bare_conf.tex
%% V1.4b
%% 2015/08/26
%% by Michael Shell
%% See:
%% http://www.michaelshell.org/
%% for current contact information.
%%
%% This is a skeleton file demonstrating the use of IEEEtran.cls
%% (requires IEEEtran.cls version 1.8b or later) with an IEEE
%% conference paper.
%%
%% Support sites:
%% http://www.michaelshell.org/tex/ieeetran/
%% http://www.ctan.org/pkg/ieeetran
%% and
%% http://www.ieee.org/

%%*************************************************************************
%% Legal Notice:
%% This code is offered as-is without any warranty either expressed or
%% implied; without even the implied warranty of MERCHANTABILITY or
%% FITNESS FOR A PARTICULAR PURPOSE! 
%% User assumes all risk.
%% In no event shall the IEEE or any contributor to this code be liable for
%% any damages or losses, including, but not limited to, incidental,
%% consequential, or any other damages, resulting from the use or misuse
%% of any information contained here.
%%
%% All comments are the opinions of their respective authors and are not
%% necessarily endorsed by the IEEE.
%%
%% This work is distributed under the LaTeX Project Public License (LPPL)
%% ( http://www.latex-project.org/ ) version 1.3, and may be freely used,
%% distributed and modified. A copy of the LPPL, version 1.3, is included
%% in the base LaTeX documentation of all distributions of LaTeX released
%% 2003/12/01 or later.
%% Retain all contribution notices and credits.
%% ** Modified files should be clearly indicated as such, including  **
%% ** renaming them and changing author support contact information. **
%%*************************************************************************


% *** Authors should verify (and, if needed, correct) their LaTeX system  ***
% *** with the testflow diagnostic prior to trusting their LaTeX platform ***
% *** with production work. The IEEE's font choices and paper sizes can   ***
% *** trigger bugs that do not appear when using other class files.       ***                          ***
% The testflow support page is at:
% http://www.michaelshell.org/tex/testflow/



\documentclass[conference, compsoc]{IEEEtran}
% Some Computer Society conferences also require the compsoc mode option,
% but others use the standard conference format.
%
% If IEEEtran.cls has not been installed into the LaTeX system files,
% manually specify the path to it like:
% \documentclass[conference]{../sty/IEEEtran}

% Some very useful LaTeX packages include:
% (uncomment the ones you want to load)


% *** MISC UTILITY PACKAGES ***
%
%\usepackage{ifpdf}
% Heiko Oberdiek's ifpdf.sty is very useful if you need conditional
% compilation based on whether the output is pdf or dvi.
% usage:
% \ifpdf
%   % pdf code
% \else
%   % dvi code
% \fi
% The latest version of ifpdf.sty can be obtained from:
% http://www.ctan.org/pkg/ifpdf
% Also, note that IEEEtran.cls V1.7 and later provides a builtin
% \ifCLASSINFOpdf conditional that works the same way.
% When switching from latex to pdflatex and vice-versa, the compiler may
% have to be run twice to clear warning/error messages.


% *** CITATION PACKAGES ***
%
%\usepackage{cite}
\usepackage{cite}
\usepackage[labelsep=period]{caption}
\usepackage{balance}
\usepackage{graphicx}
\usepackage{multirow}
\usepackage{booktabs}
%\bibliographystyle{ieeetr}
%\usepackage[sorting=none]{biblatex}
%\bibliography{journals,phd-references}
% cite.sty was written by Donald Arseneau
% V1.6 and later of IEEEtran pre-defines the format of the cite.sty package
% \cite{} output to follow that of the IEEE. Loading the cite package will
% result in citation numbers being automatically sorted and properly
% "compressed/ranged". e.g., [1], [9], [2], [7], [5], [6] without using
% cite.sty will become [1], [2], [5]--[7], [9] using cite.sty. cite.sty's
% \cite will automatically add leading space, if needed. Use cite.sty's
% noadjust option (cite.sty V3.8 and later) if you want to turn this off
% such as if a citation ever needs to be enclosed in parenthesis.
% cite.sty is already installed on most LaTeX systems. Be sure and use
% version 5.0 (2009-03-20) and later if using hyperref.sty.
% The latest version can be obtained at:
% http://www.ctan.org/pkg/cite
% The documentation is contained in the cite.sty file itself.






% *** GRAPHICS RELATED PACKAGES ***
%
\ifCLASSINFOpdf
  %\usepackage[pdftex]{graphicx}
  % declare the path(s) where your graphic files are
  %\graphicspath{{pdf/}{jpeg/}}
  % and their extensions so you won't have to specify these with
  % every instance of \includegraphics
  % \DeclareGraphicsExtensions{.pdf,.jpeg,.png}
  %\DeclareGraphicsExtensions{.pdf,.jpeg,.png}
\else
  % or other class option (dvipsone, dvipdf, if not using dvips). graphicx
  % will default to the driver specified in the system graphics.cfg if no
  % driver is specified.
  % \usepackage[dvips]{graphicx}
  % declare the path(s) where your graphic files are
  % \graphicspath{{../eps/}}
  % and their extensions so you won't have to specify these with
  % every instance of \includegraphics
  % \DeclareGraphicsExtensions{.eps}
  %\usepackage{subfigure}
\fi
% graphicx was written by David Carlisle and Sebastian Rahtz. It is
% required if you want graphics, photos, etc. graphicx.sty is already
% installed on most LaTeX systems. The latest version and documentation
% can be obtained at: 
% http://www.ctan.org/pkg/graphicx
% Another good source of documentation is "Using Imported Graphics in
% LaTeX2e" by Keith Reckdahl which can be found at:
% http://www.ctan.org/pkg/epslatex
%
% latex, and pdflatex in dvi mode, support graphics in encapsulated
% postscript (.eps) format. pdflatex in pdf mode supports graphics
% in .pdf, .jpeg, .png and .mps (metapost) formats. Users should ensure
% that all non-photo figures use a vector format (.eps, .pdf, .mps) and
% not a bitmapped formats (.jpeg, .png). The IEEE frowns on bitmapped formats
% which can result in "jaggedy"/blurry rendering of lines and letters as
% well as large increases in file sizes.
%
% You can find documentation about the pdfTeX application at:
% http://www.tug.org/applications/pdftex





% *** MATH PACKAGES ***
%
\usepackage{amsmath}
\DeclareMathOperator*{\argmin}{arg\,min} % thin space, limits underneath in displays
\DeclareMathOperator*{\argmax}{arg\,max} % thin space, limits underneath in displays
\newcommand{\ineq}{%
	\mathrel{\mkern1mu\underline{\mkern-1mu\in\mkern-1mu}\mkern1mu}}
%\DeclareMathOperator*{\argmin}{argmin} % no space, limits underneath in displays
%\DeclareMathOperator{\argmin}{arg\,min} % thin space, limits on side in displays
%\DeclareMathOperator{\argmin}{argmin} % no space, limits on side in displays
% A popular package from the American Mathematical Society that provides
% many useful and powerful commands for dealing with mathematics.
%
% Note that the amsmath package sets \interdisplaylinepenalty to 10000
% thus preventing page breaks from occurring within multiline equations. Use:
%\interdisplaylinepenalty=2500
% after loading amsmath to restore such page breaks as IEEEtran.cls normally
% does. amsmath.sty is already installed on most LaTeX systems. The latest
% version and documentation can be obtained at:
% http://www.ctan.org/pkg/amsmath





% *** SPECIALIZED LIST PACKAGES ***
%
\usepackage{algorithmic}
% algorithmic.sty was written by Peter Williams and Rogerio Brito.
% This package provides an algorithmic environment fo describing algorithms.
% You can use the algorithmic environment in-text or within a figure
% environment to provide for a floating algorithm. Do NOT use the algorithm
% floating environment provided by algorithm.sty (by the same authors) or
% algorithm2e.sty (by Christophe Fiorio) as the IEEE does not use dedicated
% algorithm float types and packages that provide these will not provide
% correct IEEE style captions. The latest version and documentation of
% algorithmic.sty can be obtained at:
% http://www.ctan.org/pkg/algorithms
% Also of interest may be the (relatively newer and more customizable)
% algorithmicx.sty package by Szasz Janos:
% http://www.ctan.org/pkg/algorithmicx




% *** ALIGNMENT PACKAGES ***
%
\usepackage{array}
% Frank Mittelbach's and David Carlisle's array.sty patches and improves
% the standard LaTeX2e array and tabular environments to provide better
% appearance and additional user controls. As the default LaTeX2e table
% generation code is lacking to the point of almost being broken with
% respect to the quality of the end results, all users are strongly
% advised to use an enhanced (at the very least that provided by array.sty)
% set of table tools. array.sty is already installed on most systems. The
% latest version and documentation can be obtained at:
% http://www.ctan.org/pkg/array


% IEEEtran contains the IEEEeqnarray family of commands that can be used to
% generate multiline equations as well as matrices, tables, etc., of high
% quality.




% *** SUBFIGURE PACKAGES ***
%\ifCLASSOPTIONcompsoc
%  \usepackage[caption=false,font=normalsize,labelfont=sf,textfont=sf]{subfig}
%\else
%  \usepackage[caption=false,font=footnotesize]{subfig}
%\fi
% subfig.sty, written by Steven Douglas Cochran, is the modern replacement
% for subfigure.sty, the latter of which is no longer maintained and is
% incompatible with some LaTeX packages including fixltx2e. However,
% subfig.sty requires and automatically loads Axel Sommerfeldt's caption.sty
% which will override IEEEtran.cls' handling of captions and this will result
% in non-IEEE style figure/table captions. To prevent this problem, be sure
% and invoke subfig.sty's "caption=false" package option (available since
% subfig.sty version 1.3, 2005/06/28) as this is will preserve IEEEtran.cls
% handling of captions.
% Note that the Computer Society format requires a larger sans serif font
% than the serif footnote size font used in traditional IEEE formatting
% and thus the need to invoke different subfig.sty package options depending
% on whether compsoc mode has been enabled.
%
% The latest version and documentation of subfig.sty can be obtained at:
% http://www.ctan.org/pkg/subfig




% *** FLOAT PACKAGES ***
%
%\usepackage{fixltx2e}
% fixltx2e, the successor to the earlier fix2col.sty, was written by
% Frank Mittelbach and David Carlisle. This package corrects a few problems
% in the LaTeX2e kernel, the most notable of which is that in current
% LaTeX2e releases, the ordering of single and double column floats is not
% guaranteed to be preserved. Thus, an unpatched LaTeX2e can allow a
% single column figure to be placed prior to an earlier double column
% figure.
% Be aware that LaTeX2e kernels dated 2015 and later have fixltx2e.sty's
% corrections already built into the system in which case a warning will
% be issued if an attempt is made to load fixltx2e.sty as it is no longer
% needed.
% The latest version and documentation can be found at:
% http://www.ctan.org/pkg/fixltx2e


%\usepackage{stfloats}
% stfloats.sty was written by Sigitas Tolusis. This package gives LaTeX2e
% the ability to do double column floats at the bottom of the page as well
% as the top. (e.g., "\begin{figure*}[!b]" is not normally possible in
% LaTeX2e). It also provides a command:
%\fnbelowfloat
% to enable the placement of footnotes below bottom floats (the standard
% LaTeX2e kernel puts them above bottom floats). This is an invasive package
% which rewrites many portions of the LaTeX2e float routines. It may not work
% with other packages that modify the LaTeX2e float routines. The latest
% version and documentation can be obtained at:
% http://www.ctan.org/pkg/stfloats
% Do not use the stfloats baselinefloat ability as the IEEE does not allow
% \baselineskip to stretch. Authors submitting work to the IEEE should note
% that the IEEE rarely uses double column equations and that authors should try
% to avoid such use. Do not be tempted to use the cuted.sty or midfloat.sty
% packages (also by Sigitas Tolusis) as the IEEE does not format its papers in
% such ways.
% Do not attempt to use stfloats with fixltx2e as they are incompatible.
% Instead, use Morten Hogholm'a dblfloatfix which combines the features
% of both fixltx2e and stfloats:
%
% \usepackage{dblfloatfix}
% The latest version can be found at:
% http://www.ctan.org/pkg/dblfloatfix


\usepackage[english]{babel}
\usepackage{blindtext}

% *** PDF, URL AND HYPERLINK PACKAGES ***
%
\usepackage{url}
% url.sty was written by Donald Arseneau. It provides better support for
% handling and breaking URLs. url.sty is already installed on most LaTeX
% systems. The latest version and documentation can be obtained at:
% http://www.ctan.org/pkg/url
% Basically, \url{my_url_here}.




% *** Do not adjust lengths that control margins, column widths, etc. ***
% *** Do not use packages that alter fonts (such as pslatex).         ***
% There should be no need to do such things with IEEEtran.cls V1.6 and later.
% (Unless specifically asked to do so by the journal or conference you plan
% to submit to, of course. )


% correct bad hyphenation here
\hyphenation{op-tical networks semi-conduc-tor}
\hyphenation{PASCAL}

\begin{document}
%
% paper title
% Titles are generally capitalized except for words such as a, an, and, as,
% at, but, by, for, in, nor, of, on, or, the, to and up, which are usually
% not capitalized unless they are the first or last word of the title.
% Linebreaks \\ can be used within to get better formatting as desired.
% Do not put math or special symbols in the title.
\title{Batik Classification using Deep Convolutional Network Transfer Learning}
% author names and affiliations
% use a multiple column layout for up to three different
% affiliations
\author{\IEEEauthorblockN{Yohanes Gultom, Aniati Murni Arymurthy\\\\}
	\IEEEauthorblockA{
		\normalfont Faculty of Computer Science, Universitas Indonesia\\\\
		\textit{Email:{yohanes.gultom@ui.ac.id}} 
	} \\
}

% conference papers do not typically use \thanks and this command
% is locked out in conference mode. If really needed, such as for
% the acknowledgment of grants, issue a \IEEEoverridecommandlockouts
% after \documentclass

% for over three affiliations, or if they all won't fit within the width
% of the page, use this alternative format:
% 
%\author{\IEEEauthorblockN{Michael Shell\IEEEauthorrefmark{1},
%Homer Simpson\IEEEauthorrefmark{2},
%James Kirk\IEEEauthorrefmark{3}, 
%Montgomery Scott\IEEEauthorrefmark{3} and
%Eldon Tyrell\IEEEauthorrefmark{4}}
%\IEEEauthorblockA{\IEEEauthorrefmark{1}School of Electrical and Computer Engineering\\
%Georgia Institute of Technology,
%Atlanta, Georgia 30332--0250\\ Email: see http://www.michaelshell.org/contact.html}
%\IEEEauthorblockA{\IEEEauthorrefmark{2}Twentieth Century Fox, Springfield, USA\\
%Email: homer@thesimpsons.com}
%\IEEEauthorblockA{\IEEEauthorrefmark{3}Starfleet Academy, San Francisco, California 96678-2391\\
%Telephone: (800) 555--1212, Fax: (888) 555--1212}
%\IEEEauthorblockA{\IEEEauthorrefmark{4}Tyrell Inc., 123 Replicant Street, Los Angeles, California 90210--4321}}

% use for special paper notices
%\IEEEspecialpapernotice{(Invited Paper)}
% make the title area
%\maketitle
\twocolumn[
{\csname @twocolumnfalse\endcsname \maketitle}
{\csname @twocolumnfalse\endcsname 
	%\input{abstract}
	\renewcommand{\abstractname}{Abstract}
	\begin{abstract}
		\noindent
		\normalfont 
		Batik fabric is one of the most profound cultural heritage in Indonesia. Hence, continuous research on understanding it is necessary to preserve it. Despite of being one of the most common research task, Batik's pattern automatic classification still requires some improvement especially in regards to invariance dilemma. Convolutional neural network (ConvNet) is one of deep learning architecture which able to learn data representation by combining local receptive inputs, weight sharing and convolutions in order to solve invariance dilemma in image classification. Using dataset of 523 photos (5 classes), the experiments show that the proposed model, which used deep ConvNet VGG16 as feature extractor (transfer learning), achieved average of 0.69 $\pm$ 0.06 accuracy while the best SIFT and SURF-based models achieved 0.67 $\pm$ 0.14 and 0.68 $\pm$ 0.14 respectively. 
		\\\\
		\noindent
		\textbf{Keywords}: \textit{Batik, classification, deep learning, transfer learning} \\\\
	\end{abstract}
	
	\renewcommand{\abstractname}{Abstrak}
	\begin{abstract}
		\noindent
		\normalfont 
		Kain Batik adalah salah satu warisan kebudayaan Indonesia yang sangat berharga. Oleh karena itu, penelitian yang berkesinambungan perlu dilakukan untuk melestarikannya. Sekalipun telah menjadi topik penelitian yang umum, klasifikasi pola Batik secara otomatis masih memiliki beberapa tantangan yang perlu diselesaikan. Salah satu tantangan tersebut adalah masalah \textit{invariance dilemma}. \textit{Convolutional neural network} (ConvNet) adalah salah satu arsitektur \textit{deep learning} yang mampu mempelajari representasi data dengan mengkombinasikan teknik \textit{local receptive inputs}, \textit{weight sharing} dan \textit{convolutions} untuk mengatasi masalah \textit{invariance dilemma} pada klasifikasi citra seperti pola Batik. Eksperimen menggunakan \textit{dataset} 523 foto Batik (5 kelas) menunjukkan bahwa model yang menggunakan ConvNet VGG16 sebagai ekstraktor fitur mencapai rata-rata akurasi 0.69 $\pm$ 0.06 sedangkan model berbasis SIFT dan SURF mencapai rata-rata 0.67 $\pm$ 0.14 and 0.68 $\pm$ 0.14.
		\\\\
		\noindent
		\textbf{Kata Kunci}: \textit{Batik, klasifikasi, deep learning, transfer learning} \\\\
	\end{abstract}
	
	%\begin{IEEEkeywords}
	%%IEEEtran, journal, \LaTeX, paper, template.
	%\end{IEEEkeywords}
}
%\vspace{1cm}
]

% For peer review papers, you can put extra information on the cover
% page as needed:
% \ifCLASSOPTIONpeerreview
% \begin{center} \bfseries EDICS Category: 3-BBND \end{center}
% \fi
%
% For peerreview papers, this IEEEtran command inserts a page break and
% creates the second title. It will be ignored for other modes.
\IEEEpeerreviewmaketitle

\section{Introduction}

Batik fabric is one of the most profound cultural heritage in Indonesia. Hence, continuous research on understanding it is necessary to preserve it. One of the most popular research topic is batik classification.

Since the most prominent feature of Batik is its uniquely recurring pattern (motifs), it's natural to consider it as a key to classification. To be more specific, recognition of Batik's motifs has been considered as one of the most successful technique in Batik classification especially using Scale-Invariant Feature Transform (SIFT) \cite{azhar2015batik} \cite{nurhaida2015automatic} and Speeded up robust features (SURF) \cite{willy2013evaluation}. Classifications using other features such as color and contrast are showing potentials but need to be researched further \cite{moertini2005algorithms}.

Deep learning based models have outperformed state-of-the-art methods in many domains including image classification and object recognition \cite{lecun2015deep}. One of the deep learning models, convolutional neural network (convnet) \cite{lecun1998gradient}, is currently considered as the state-of-the-art of image classification model as it was used as the base structure by ILSVRC-2014 top achievers \cite{simonyan2014very}. Therefore convnet may also be used to improve result on other image classification problems such as Batik classification.

In this paper, we propose a neural network Batik classification model that uses pre-trained deep convolutional network (VGG16) \cite{simonyan2014very} as a feature extractor. Features from a dataset of five general classes of Indonesian Batik (Parang, Lereng, Kawung, Ceplok and Nitik) are extracted using VGG16 and used to test various configuration of neural networks in order to find the best model. To test the capability of the model to solve invariance dilemma, some tests are also done on the rotated and zoomed version of the dataset. Finally, the proposed model is compared with SIFT and SURF based models using the same dataset.

\section{Related Works}

\begin{figure}[h]
	\begin{center}
		\includegraphics[width=2.0in]{../sift-keypoint}
		\renewcommand{\figurename}{Fig.}		
		\caption{SIFT Keypoint}
		\label{fig_keypoint}
	\end{center}
\end{figure}

\begin{figure}[h]
	\begin{center}
		\includegraphics[width=2.0in]{../batik-parang-keypoints}
		\renewcommand{\figurename}{Fig.}		
		\caption{SIFT keypoints in Batik Parang}
		\label{fig_batik_parang_keypoints}
	\end{center}
\end{figure}

Recent researches in Batik classification can be divided into two groups: (1) Researches on classification using handcrafted features (eg. SIFT and SURF), and (2) researches on classification using automatically extracted features using deep learning.

\subsection{Classification using Handcrafted Features}

\begin{figure}[h]
	\begin{center}
		\includegraphics[width=3.0in]{../sift-bag-of-words}
		\renewcommand{\figurename}{Fig.}	
		\caption{SIFT for building bag of words visual vocabularies}
		\label{fig_sift_bag_of_words}
	\end{center}
\end{figure}

One of the initial work on Batik Classification was done using Generalized Hough Transform (GHT) 

Since Batik classification has been researched for quite some time, current available methods are robust enough to noise addition, compression, and retouching of the input images. However most of them are still having difficulties with variance in transformations which involve either translation, rotation, scaling or combinations of them \cite{nurhaida2015automatic}. Recent improvements on Batik classification were motivated by the emergence of Scale-Invariant Feature Transform (SIFT)\cite{lowe2004distinctive} and Speeded up robust features (SURF)\cite{bay2006surf}. Both of these keypoint-based feature extraction methods are proposed to solve the transformation invariance dilemma.

SIFT keypoint is a circular image region with an orientation which can be obtained by detecting extrema of Difference of Gaussian (DoG) pyramid \cite{lowe2004distinctive}. It's defined by four parameters: center coordinates x and y, scale and its orientation (an angle expressed in radians) as shown in Figure \ref{fig_keypoint}. An image, for example Batik image, may contains multiple keypoints as shown in Figure \ref{fig_keypoint}. In order to be efficiently and effectively used as a feature for classification, the keypoint need to be represented as SIFT descriptor. By definition it is a 3-dimensional spatial histogram of the image gradients characterizing a SIFT keypoint.

Recent research \cite{nurhaida2015automatic} proved that using SIFT descriptors to calculate similarity between Batik images can give 91.53\% accuracy. Voting Hough Transform was applied to the descriptors to eliminate mismatched keypoint candidates. This research suggested that the original SIFT descriptor matching shouldn't be directly used to calculate similarity of Batik images due to many numbers of mismatched keypoints.

Another research \cite{azhar2015batik} proposed a classification method using support vector machine (SVM) fed by bag of words (BOF) features extracted using SIFT descriptors. In this research, SIFT descriptors also weren't used directly as features for SVM but were clustered using k-means vector quantization algorithm to build vocabularies. These visual vocabularies then used to describe each images and fed to SVM classifier. The experiment results showed very good average accuracy of 97.67\% for normal images, 95.47\% for rotated images and 79\% for scaled images. Besides that SIFT and bag of words made a good feature extractor, this research also concluded that further works need to handle scaled Batik image cases.

An earlier research \cite{willy2013evaluation} proved that SURF can extract transformation invariant features faster than SIFT for classification of Songket, another Indonesian traditional fabric with motifs just like Batik. Unlike the others, this research used SIFT and SURF features directly to compute the matching scores between Songket images. The scores are calculated by (1) the number of matched keypoints and (2) the average total distance of the n-nearest keypoints. The result of experiments showed that the matching accuracy with SIFT features was 92-100\% and 65-97\% with SURF. With SURF features, the accuracy dropped quite significant if salt and pepper noises were added while SIFT was more stable. Apparently, this one wasn't paying much attention to transformation variance as it didn't apply transformation noise as in other research\cite{azhar2015batik}.


\subsection{Classification using Deep Learning}

Deep learning is a multilayer representation learning in artificial neural network \cite{lecun2015deep}. While representation learning itself is a method in machine learning to automatically extract/learn representation (features) from raw data. The representation of the raw data then can be used for recognition or classification task. Some fundamental deep learning architectures for instances are convolutional neural network (ConvNet), deep belief network (DBN), autoencoder (AE) and recurrent neural network (RNN). Despite of being an old idea, it was recently emerged due to the several factors: (1) discovery of new techniques (eg. pretraining \& dropout) and new activation functions (eg. ReLU), (2) enormous supply of data (big data), and (3) rapid improvement in computational hardware, especially GPU.

Although not yet many, the advent of deep learning also motivated a research on Batik classification using convolutional stacked autoencoder \cite{menzata2014sistem}. This research proposed the usage of convolutional transformations to reduce the input nodes of stacked autoencoder. The experiment showed that this deep architecture was able to achieve 81,73\% accuracy by using small patches of Batik for training. When noises were added its accuracy dropped to 49\% for gaussian noises, 61\% for rotations, 70\% for scalings and 75\% for illumination noises. Another research have shown that deep architecture such as convolutional neural network should be able to outperform handcrafted features such as SIFT\cite{fischer2014descriptor}. Therefore further research on Batik classification using deep learning architectures is encouraged.

\section{Methodology}

We propose a deep convolutional neural network composed by a pre-trained VGG16 (without its top layer) as automatic feature extractor and a fully-connected tangent hiperbolic (tanh) feed-forward neural network as classifier. The method of using pre-trained deep network as part of another neural network to solve different (but related) task can be considered as transfer learning or self-taught learning \cite{raina2007self}.

\subsection{Convolutional Neural Network}

\begin{figure}[h]
	\begin{center}
		\includegraphics[width=3.0in]{../lenet5}
		\renewcommand{\figurename}{Fig.}		
		\caption{LeNet5 convolutional network}
		\label{fig_lenet5_convnet}
	\end{center}
\end{figure}

Convolutional network is a special kind of neural network optimized to learn representation of an image \cite{lecun2015deep}. It introduces 2 new types of hidden layers: convolutional and subsampling/pooling layers. Each layer in convnet connects neurons (pixels) from their input layer in form of local receptives (square patches) through a shared weights to a feature map \cite{lecun1998gradient}. On top of a set of convolutional and pooling layers, some fully-connected layers are added as classifier as described by Figure \ref{fig_lenet5_convnet}.

Our proposed model uses 5 set of convolutional and pooling layers using rectified unit (ReLu) activation function following example of VGG16 as shown in Figure \ref{fig_vgg16}. The differences are our model uses 2 fully-connected hyperbolic tangent (tanh) (Equation \ref{eq_tanh}) activated layers as classifier (instead of ReLu) and a SoftMax (Equation \ref{eq_softmax}) layer as an output. We also uses Dropout regularization after each tanh fully-connected layers to avoid overfitting by randomly drop/turn off (set value to zero) hidden nodes (Equation \ref{eq_dropout}) \cite{srivastava2014dropout}. The whole architecture is shown in Figure \ref{fig_proposed_method}.

\begin{equation}
y_{i}={\frac {2}{1+e^{-2x _{i}}}}-1
\label{eq_tanh}
\end{equation}

\begin{equation}
y_{i}={\frac {e^{x _{i}}}{\sum _{k=1}^K e^{x _{k}}}} \quad \text{, for i=1..K}
\label{eq_softmax}
\end{equation}

\begin{align}
r_{j}^{x} \sim Bernoulli(p), \nonumber \\
\tilde{y} _{i}= r _{i} * y _{i}
\label{eq_dropout}
\end{align}


\begin{figure}[h]
	\begin{center}
		\includegraphics[width=3.0in]{../vgg16}
		\renewcommand{\figurename}{Fig.}		
		\caption{VGG16 deep convolutional network model of Visual Geometry Group, Oxford}
		\label{fig_vgg16}
	\end{center}
\end{figure}

\begin{figure}[h]
	\begin{center}
		\includegraphics[width=3.0in]{../proposed_method}
		\renewcommand{\figurename}{Fig.}		
		\caption{Proposed method composed of pre-trained VGG16 as extractor and fully-connected TanH and Softmax as classifier}
		\label{fig_proposed_method}
	\end{center}
\end{figure}

\subsection{Transfer Learning}

Deep neural networks usually requires a lot of training data in order to learn the representation of the data. In case there is not enough training data, there are several techniques to help neural networks model learns data representation using small training data. One of the technique is transferring knowledge of other pre-trained neural network model to our model. This technique is known as transfer learning or self-taught learning \cite{raina2007self}.

Our proposed model uses transferred knowledge (layer weights) from pre-trained VGG16 model provided by deep learning framework Keras \footnote{\url{https://keras.io/applications/#vgg16}} which was pre-trained using 1,000,000 images dataset from ImageNet. We use VGG16 bottom layers weights (excluding the fully-connected layers) to initialize our model weights and train the classifier part only. This method allows us to shorten the time needed to train our model. Moreover, training the model using GPU improve the speed even more.

To improve comprehension and reproducibility, we  publish our model Python code in public online code repository
\footnote{\url{https://github.com/yohanesgultom/deep-learning-batik-classification}}. 
We also use opensource Theano-backed Keras as deep learning framework and Scikit-Learn\footnote{\url{http://scikit-learn.org/}} as model evaluation framework to improve reusability.

\section{Experiments and Results}

In order to measure the performance of our model, we trained our model and compared it with SIFT and SURF based models.

\subsection{Experiments}

The dataset was used in this research is a Batik dataset compiled by Machine Learning and Computer Vision (MLCV) Lab, Faculty of Computer Science, University of Indonesia. This dataset consists of 603 Batik photos ($\pm$ 78.3 MB) gathered from various sources thus having different size, quality and view angle.

We also tried to tune the dataset by deleting some duplicated (similar photos under a class) and conflicting photos (same photos exist under different classes). The tuned dataset has 523 photos left ($\pm$ 68.7 MB) and considered as separate dataset in this experiment.

All classifiers in this experiment were trained using 553 Batik photos (around 52-169 per class) and tested using 50 photos (10 classes per class) from regular dataset (9:1 ratio). The classifiers were also trained using 476 photos and tested with 47 photos from tuned dataset in order to observe the difference.

Our neural network classifier was trained for 50 epoch using Cross Entropy function to calculate loss (Equation \ref{eq_cross_entropy_loss}) and optimized using Stochastic Gradient Descent (SGD) (Equation \ref{eq_sgd}) to update weights.

\begin{equation}
V(f(\vec{x}),t) = -t\ln(f(\vec{x}))-(1-t)\ln(1-f(\vec{x}))
\label{eq_cross_entropy_loss}
\end{equation}

\begin{equation}
w:=w-\eta \nabla Q_{i}(w)+\alpha \Delta w
\label{eq_sgd}
\end{equation}

The SIFT and SURF models were trained in similar manner using similar methods described in related research \cite{azhar2015batik} and also illustrated in Figure \ref{fig_sift_bag_of_words}:

\begin{enumerate}
	\item Image descriptors were extracted according to their feature extractor (SIFT or SURF)
	\item Descriptors were clustered to 5 clusters using K-Means to get visual vocabularies for Bag of Words (BoW)
	\item Those 5 visual vocabularies then used to compute BoW features from SIFT/SURF image descriptors
	\item Finally a multi-class SVM classifier were trained using the BoW features
\end{enumerate}

All experiments were conducted using Intel Core i7-5960X CPU, 66 GB RAM, NVIDIA GTX 980 4GB GPU, 240GB SSD, Debian 8 OS. The proposed model ran on GPU while SIFT/SURF SVM models ran on CPU.

\subsection{Results}

As shown in Figure \ref{fig_accuracy_comparison}, the proposed model outperformed SIFT and SURF based classifier by 24\% and 34\% respectively when using regular dataset. Moreover by using tuned dataset, our model outperformed other models by larger margin 34\% and 49\%.

\begin{figure*}[h]
	\begin{center}
		\includegraphics[width=\textwidth]{../chart_accuracy_rev}
		\renewcommand{\figurename}{Fig.}
		\caption{Models accuracy comparison}
		\label{fig_accuracy_comparison}
	\end{center}
\end{figure*}

\begin{table*}
	\begin{center}
		\caption{Performance comparison of fully-connected layers configurations}
		\label{tab_experiment_results}	
		\begin{tabular}{lllll}
		    \toprule
			\multirow{2}{*}[-3pt]{Hidden Layer(s)}  & \multicolumn{4}{c}{Number of Epochs} \\ 
		    \cmidrule{2-5} 
		  & 50 & 100 & 200 & 300 \\ 
		     \midrule
2 Sigmoid & 0.62 $\pm$ 0.11 & 0.68 $\pm$ 0.1 & 0.68 $\pm$ 0.08 & 0.68 $\pm$ 0.12 \\
2 ReLU & 0.67 $\pm$ 0.13 & 0.67 $\pm$ 0.07 & 0.69 $\pm$ 0.1 & 0.67 $\pm$ 0.12 \\
2 TanH & 0.67 $\pm$ 0.11 & \textbf{0.69 $\pm$ 0.06} & 0.67 $\pm$ 0.11 & 0.68 $\pm$ 0.11 \\
2 ReLU + TanH & 0.68 $\pm$ 0.1 & 0.68 $\pm$ 0.08 & 0.69 $\pm$ 0.08 & 0.67 $\pm$ 0.08 \\
1 Sigmoid & 0.67 $\pm$ 0.07 & 0.69 $\pm$ 0.13 & 0.67 $\pm$ 0.08 & 0.68 $\pm$ 0.13 \\
1 ReLU  & 0.67 $\pm$ 0.07 & 0.63 $\pm$ 0.05 & 0.65 $\pm$ 0.07 & 0.56 $\pm$ 0.19 \\
1 TanH & 0.69 $\pm$ 0.07 & 0.67 $\pm$ 0.11 & 0.66 $\pm$ 0.11 & 0.66 $\pm$ 0.08 \\
		    \bottomrule
		\end{tabular}
	\end{center}
\end{table*} 

\begin{table*}
	\begin{center}
		\caption{Tests on zoomed and rotated Batik images}
		\label{tab_experiment_results_2}	
		\begin{tabular}{llllll}
		    \toprule
			\multirow{2}{*}[-3pt]{Transformation}  & \multicolumn{4}{c}{Experiment No} & \multirow{2}{*}[-3pt]{Average} \\ 
		    \cmidrule{2-4} 
		  & 1 & 2 & 3 & \\ 
		     \midrule
Zoom in 10\% & 0.76 & 0.68 & 0.74 & \textbf{0.73} \\
Zoom in 30\% & 0.68 & 0.72 & 0.72 & \textbf{0.71} \\
Zoom in 50\% & 0.72 & 0.7 & 0.62 & \textbf{0.68} \\
Rotate 90$^{\circ}$ & 0.55 & 0.47 & 0.55 & \textbf{0.52} \\
Rotate 180$^{\circ}$ & 0.64 & 0.62 & 0.63 & \textbf{0.63} \\
Rotate 270$^{\circ}$ & 0.46 & 0.49 & 0.45 & \textbf{0.47} \\
		    \bottomrule
		\end{tabular}
	\end{center}
\end{table*} 


\section{Conclusion and Future Works}

As shown by the experiments, our model, which is based on deep convolutional network, outperformed SIFT and SURF based models in term of accuracy as well as processing time. This confirms that automatic feature extraction using pre-trained convolutional are able to handle transformation invariant features such as Batik motifs better than SIFT and SURF as also concluded by related research \cite{fischer2014descriptor}.

Moreover, deep neural networks is more scalable as its training computation is composed of matrix multiplication which can be easily parallelized on GPU. Supported by keep-growing GPU hardwares, deep neural networks can always be improved in term of speed and capacity.

We also found out that current Batik dataset has a lot of room for improvements:

\begin{enumerate}
	\item Multi-labeled data. As majority of the data are mixed-motif Batik, the dataset must provide more than one labels for each applicable sample.
	\item Clearly distinguished samples between classes. For instance, Parang and Lereng motifs data often overlaps each other. This condition often confuses classifier during training and causes less accurate generalization.
	\item Homogeneous quality of data. Due to the various sources of data, the quality (resolution, size, view angle) of the data are also various. Removing low quality data and preprocessing high quality ones may produce homogeneous data and improve classifier training process.
\end{enumerate}


% An example of a floating figure using the graphicx package.
% Note that \label must occur AFTER (or within) \caption.
% For figures, \caption should occur after the \includegraphics.
% Note that IEEEtran v1.7 and later has special internal code that
% is designed to preserve the operation of \label within \caption
% even when the captionsoff option is in effect. However, because
% of issues like this, it may be the safest practice to put all your
% \label just after \caption rather than within \caption{}.
%
% Reminder: the "draftcls" or "draftclsnofoot", not "draft", class
% option should be used if it is desired that the figures are to be
% displayed while in draft mode.
%
%\begin{figure}[!t]
%\centering
%\includegraphics[width=2.5in]{myfigure}
% where an .eps filename suffix will be assumed under latex, 
% and a .pdf suffix will be assumed for pdflatex; or what has been declared
% via \DeclareGraphicsExtensions.
%\caption{Simulation results for the network.}
%\label{fig_sim}
%\end{figure}

% Note that the IEEE typically puts floats only at the top, even when this
% results in a large percentage of a column being occupied by floats.
\balance

% An example of a double column floating figure using two subfigures.
% (The subfig.sty package must be loaded for this to work.)
% The subfigure \label commands are set within each subfloat command,
% and the \label for the overall figure must come after \caption.
% \hfil is used as a separator to get equal spacing.
% Watch out that the combined width of all the subfigures on a 
% line do not exceed the text width or a line break will occur.
%
%\begin{figure*}[!t]
%\centering
%\subfloat[Case I]{\includegraphics[width=2.5in]{box}%
%\label{fig_first_case}}
%\hfil
%\subfloat[Case II]{\includegraphics[width=2.5in]{box}%
%\label{fig_second_case}}
%\caption{Simulation results for the network.}
%\label{fig_sim}
%\end{figure*}
%
% Note that often IEEE papers with subfigures do not employ subfigure
% captions (using the optional argument to \subfloat[]), but instead will
% reference/describe all of them (a), (b), etc., within the main caption.
% Be aware that for subfig.sty to generate the (a), (b), etc., subfigure
% labels, the optional argument to \subfloat must be present. If a
% subcaption is not desired, just leave its contents blank,
% e.g., \subfloat[].


% An example of a floating table. Note that, for IEEE style tables, the
% \caption command should come BEFORE the table and, given that table
% captions serve much like titles, are usually capitalized except for words
% such as a, an, and, as, at, but, by, for, in, nor, of, on, or, the, to
% and up, which are usually not capitalized unless they are the first or
% last word of the caption. Table text will default to \footnotesize as
% the IEEE normally uses this smaller font for tables.
% The \label must come after \caption as always.
%
%\begin{table}[!t]
%% increase table row spacing, adjust to taste
%\renewcommand{\arraystretch}{1.3}
% if using array.sty, it might be a good idea to tweak the value of
% \extrarowheight as needed to properly center the text within the cells
%\caption{An Example of a Table}
%\label{table_example}
%\centering
%% Some packages, such as MDW tools, offer better commands for making tables
%% than the plain LaTeX2e tabular which is used here.
%\begin{tabular}{|c||c|}
%\hline
%One & Two\\
%\hline
%Three & Four\\
%\hline
%\end{tabular}
%\end{table}


% Note that the IEEE does not put floats in the very first column
% - or typically anywhere on the first page for that matter. Also,
% in-text middle ("here") positioning is typically not used, but it
% is allowed and encouraged for Computer Society conferences (but
% not Computer Society journals). Most IEEE journals/conferences use
% top floats exclusively. 
% Note that, LaTeX2e, unlike IEEE journals/conferences, places
% footnotes above bottom floats. This can be corrected via the
% \fnbelowfloat command of the stfloats package.

% trigger a \newpage just before the given reference
% number - used to balance the columns on the last page
% adjust value as needed - may need to be readjusted if
% the document is modified later
%\IEEEtriggeratref{8}
% The "triggered" command can be changed if desired:
%\IEEEtriggercmd{\enlargethispage{-5in}}

% references section

% can use a bibliography generated by BibTeX as a .bbl file
% BibTeX documentation can be easily obtained at:
% http://mirror.ctan.org/biblio/bibtex/contrib/doc/
% The IEEEtran BibTeX style support page is at:
% http://www.michaelshell.org/tex/ieeetran/bibtex/
%\bibliographystyle{IEEEtran}
% argument is your BibTeX string definitions and bibliography database(s)
%\bibliography{IEEEabrv,../bib/paper}
%
% <OR> manually copy in the resultant .bbl file
% set second argument of \begin to the number of references
% (used to reserve space for the reference number labels box)

\bibliographystyle{IEEEtran}
% argument is your BibTeX string definitions and bibliography database(s)
\nocite{}
\bibliography{IEEEabrv,../laporan}




% that's all folks
\end{document}


